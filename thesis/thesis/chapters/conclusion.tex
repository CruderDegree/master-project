\chapter{Conclusion}

In an attempt to discourage ASLA from building structures with isolated atoms, a representation of ASLA structures was designed. This representation was based on the bonds of each atom in the structure, namely the number and type of atoms the atom was bonded to. This representation was shown to work in figure \ref{fig:bonding_demo}, where a single atom was moved across a colorized structure, which demonstrated how atoms around it had their features updated as the atom came within or left bonding distance. \\

Using the assumption from \eqref{eq:Local_energies}, it was demonstrated that the structure representation could be used to predict energies of new structures based on features of structures already built by ASLA. The ability to predict energies is illustrated on figure \ref{fig:energy_prediction}, where the estimated energy was plotted against the DFT energy, with a resulting $R^2$-value of $\approx 0.83$. This was then used to implement a filter, where the estimated energy of proposed structures built by ASLA were used in an attempt to filter away structures containing isolated atoms. The relative reduction in structures containing isolated atoms among the proposed builds were reduced by $\SI{72.3 \pm 5.8}{\percent}$ by choosing the half of the structures with the lowest estimated energies. \\

This filtering method was then implemented into ASLA by introducing a new building policy, $\epsilon$-loop. The performance of these new agents were compared to that of the unaltered baseline agents in figure \ref{fig:S-curve}. The results indicate that the implementation hinders the structure optimization during the early episodes while not increasing the fraction of agents that find the global minimum compared to the baseline agents. Among the reasons that explain this result are that the energy predictions are a great deal more uncertain during early episodes which makes it harder to filter away bad structures. It is also suspected that taking atoms too close to each other into account in the structure representation would have had positive effect on the performance.